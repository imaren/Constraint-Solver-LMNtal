\chapter{設計}
\section{今回書いたソルバの概要}
事実 (fact) を宣言し,入力した質問 (question) の真偽を判定する.この時,事実が充足不可能なら必ず真を返す.\\
$\sim$  全ての変数について,事実 $\implies$ 質問 を判定する.
\begin{itemize}
    \item 扱う制約は,自然数,変数,$+$,$=$,($\leq$) からなる.
\end{itemize}

\begin{table}[h]
    \centering
    \small
    \begin{tabular}{ccc|c|cr}
        $t$ & : & $\mathbb{N}$ & var & $t + t$ & (term)
    \end{tabular}
\end{table}

\begin{table}[h]
    \centering
    \small
    \begin{tabular}{ccc|cr}
        $af$ & : & ($t \leq t$) & $t=t$ & (atomic formula)
    \end{tabular}
\end{table}

\begin{table}[h]
    \centering
    \small
    \begin{tabular}{ccc|cr}
        $f$ & : & $af$ &$f \wedge f$ & (formula)
    \end{tabular}
\end{table}

\section{実装}
今回の実装では,変数はハイパーリンクで表現して,同じハイパーリンクに属する場合は同じ値を持つということを表現している.
例えば,
\begin{center}
    facts: $ x_1 = x_2 \wedge x_3 = x_4$\\
\end{center}
が与えられた場合,$x_1 と x_2$,$x_3 と x_4$ はそれぞれ同じハイパーリンクに属する.
また,数値が決定できる場合は,$ n $ というアトムをかませて,int 型の値をハイパーリンクに登録する.
\begin{lstlisting}
    hh_eq@@ eq(!X,!Y) :- !X >< !Y.
    hi_eq@@ eq(!X,N) :- int(N) | n(N, !X).
    ih_eq@@ eq(N,!X) :- int(N) | n(N, !X).
\end{lstlisting}

同じハイパーリンク内に異なる2つの自然数が存在する場合は false.
\begin{lstlisting}
    n(N,!X), n(M,!X) :- N =\= M | antecedent(false).
\end{lstlisting}

異なる2つのハイパーリンクが同じ自然数を持っていた場合は併合する.
\begin{lstlisting}
    n(N,!X) \ n(M,!Y) :- N =:= M | !X >< !Y.
\end{lstlisting}

decision procedures p.86\\
等式だけではなくて,未解釈の関数についても同じように併合している.
足し算も式のまま併合するべき?

\textcolor{red}{ここまでは普通に見えるか意見が頂きたいところ}

実装の理想形:性質(公理?)をそのまま書く\\
pertial order constraint 
\begin{lstlisting}
    leq_duplicate@@ leq(!X,!Y), leq(!X,!Y) :- leq(!X,!Y).
    leq_reflexivity@@ leq(!X,!X):- .
    leq_antisymmetry@@ leq(!X,!Y), leq(!Y,!X) :- eq(!X,!Y).
    leq_transitivity@@ leq(!X,!Y), leq(!Y,!Z)\ :-uniq(!X,!Y,!Z) | leq(!X,!Z).   
\end{lstlisting}
とか,
\begin{lstlisting}
    less_duplicate@@ less(!X,!Y) \ less(!X,!Y) :- .
    less_reflexivity@@ less(!X,!X):- false.
    less_antisymmetry@@ less(!X,!Y), less(!Y,!X) :- false.
    less_transitivity@@ less(!X,!Y), less(!Y,!Z)\ :-uniq(!X,!Y,!Z) | less(!X,!Z).
\end{lstlisting}
みたいな形.
\url{https://en.wikipedia.org/wiki/Inequality_(mathematics)}


decision procedures p.86\\
等式理論において,
\begin{center}
    $x_1 = x_2 \vee (x_2 = x_3 \wedge x_4 = x_5 \wedge x_5 \neq x_1 \wedge \hbox{\sout{$F(x_1) \neq F(x_3)$}})$
\end{center}
の論理和の部分は,ケースを分けて考えると書いてある.
→ LMNtal だったらどうやる?

例えば,
\begin{center}
    $(x_1 = x_2 \vee x_1 = x_3) \wedge (x_2 = x_4 \wedge x_1 \neq x_2)$
\end{center}
は,以下に変形できる(Disjunction Normal Form)
\begin{center}
    $(x_1 = x_2 \wedge x_2 = x_4 \wedge x_1 \neq x_2) \vee$\\
    $(x_1 = x_3 \wedge x_2 = x_4 \wedge x_1 \neq x_2)$
\end{center}
ここで,今までの方法で単に
\begin{enumerate}
    \item !X1 $><$ !X2
    \item !X2 $><$ !X4
    \item より,!X1 = !X2 = !X4
    \item !X1 = !X2 と neq(!X1,!X2) は矛盾するので 第1節が unsatisfiable.
\end{enumerate}
とすると,第2節も unsatisfiable になる(併合が残ってしまうので)

\begin{itemize}
    \item 解決策の案
\end{itemize}
DNF(Disjunction Normal Form) を仮定して,disjunction 毎に独立な変数名を宣言する.
つまり,上の式は以下のようにする.
\begin{center}
    $(x_1 = x_2 \wedge x_2 = x_4 \wedge x_1 \neq x_2) \vee$\\
    $(x_{1_1} = x_{1_3} \wedge x_{1_2} = x_{1_4} \wedge x_{1_1} \neq x_{1_2})$
\end{center}
さらに,同じ節の等式は,同じハイパーリンクに属するとする.
すると,以下のようにプログラムを書き換えると,できそう?
or の埋め込み(3節以上の or )もできるようになっている.
(できるというのは,充足可能か判定するということ.量化子はまだ考えていない←まだ読めていないが,
量化子はうまく除去できるのかな~という期待 quantifier elimination という言葉をよく見るので)
\begin{lstlisting}
eq(!X1,!X2,!H1), eq(!X2,!X3,!H1), neq(!X1,!X2,!H1).
eq(!X1_1,!X1_3,!H2), eq(!X1_2,!X1_4,!H2), neq(!X1_1,!X1_2,!H2).
ans = or(!H1,!H2).

%or
Ret = or(!X,C), unsat(!X) :- Ret = or(unsat,C).
Ret = or(C,!X), unsat(!X) :- Ret = or(C,unsat).
Ret = or(unsat,unsat):- Ret = unsat.
Ret = or(sat,C) :- ground(C) | Ret = sat.
Ret = or(C,sat) :- ground(C) | Ret = sat.

%neq
neq(!X,!Y,!H) :- !X = !Y | unsat(!H).

% one variable has 2 or more integer
n(N,!X,!H), n(M,!X,!H) :- N =\= M | unsat(!H).

% two variable has same integer
n(N,!X) \ n(M,!Y) :- N =:= M | !X >< !Y.

%equality
hh_eq@@ eq(!X,!Y,!H) :- !X >< !Y.
hi_eq@@ eq(!X,N,!H) :- int(N) | n(N,!X,!H).
ih_eq@@ eq(N,!X,!H) :- int(N) | n(N,!X,!H).

%sat
%全ての伝播が終わって,unsat でなければ sat
Ret = or(!X,C) :- ground(C) | Ret = sat.
Ret = or(C,!X) :- ground(C) | Ret = sat.
\end{lstlisting}

さっきまでは,DNFに対する標準的な処理+equality, inequality\\
ここからは,線形算術に入る.(decision procedures p.97~)

線形算術のシンタックスは以下の通り.ドメインは整数
\begin{center}
    $formula\: :\: formula\wedge formula \:|\: (formula) \:|\: atom$\\
    $atom \::\: sum \: op \: sum$\\
    $op \::\: = \:|\: \leq \:|\: <$\\
    $sum \::\: term \:|\: sum+term$\\
    $term \::\: identifier \:|\: constant \:|\: constant \: identifier$
\end{center}
$identifier$ は,整数上の変数,$constant$ は整数定数.

decision procedures p.98
Linear Arithmetic の ソルバ
%integer linear problems は 5.3 でカバー
\begin{itemize}
    \item Simplex: 数値最適化のための最も古いアルゴリズムの1つ.実数上の線形制約の論理積を与えて,目的関数の最適値を求める.
    \item general Simplex: 目的関数を与えない Simplex.充足可能かどうかを決定する.
    \item Fourier-Motzkin variable elimination: Simplex より効率は劣るが,実装が比較的容易.実数上の線形制約の論理積の充足可能性を決定.
    \item Omega test: Simplex より効率は劣るが,実装が容易.整数上の線形制約の論理積の充足可能性を決定.
\end{itemize}

omega test について考えていく.

入力は以下の形の論理積.
\begin{center}
    $ \sum_{i=1}^na_ix_i=b$ or $\sum_{i=1}^na_ix_i\leq b$
\end{center}
例えば以下のような式が当てはまる.
\begin{center}
    $ 2x_1 + 3x_2 = 10 \wedge x_1 + -1x_2\leq 0$
\end{center}
係数 $a_i$ は整数を仮定.(最小公倍数をかけて全部整数係数になるように変形する.)

手順1:equality の左辺の係数($a_1,\dots,a_n$)の最大公約数 $g$ で両辺を割る.この時,$g$ が $b$ を割り切れなかったら unsat.\\
補足:
\begin{center}
    $ \sum_{i=1}^na_ix_i=b$\\
    について\\
    $b \:mod\: gcd(a_1,\dots,a_n) = 0 \Leftrightarrow 整数解が存在する$
\end{center}


手順2:inequality の左辺の係数($a_1,\dots,a_n$)の最大公約数 $g$ で両辺を割る.右辺は小数点以下切り捨て(普通の int 型の割り算).

正規化部分.
\begin{lstlisting}
    %%mul(constant, !Identifier, !Hlink)
    %% hlink is unique for each equality
    gcd0@@ Ret = mul(N,!X,!E) \:- int(N), uniq(!X,!E) | gcd(N,!E).
    
    gcd1@@ gcd(0,!H) :- .
    gcd2@@ gcd(N,!H) \ gcd(M,!H) :- N =< M, G = M-N | gcd(G,!H).
    
    %normalization using gcd of identifiers 
    %omega test の入力の右辺は整数定数のはずなので,等式,不等式ごとに n(N,!E) が1つ繋がってるはず
    norm_eq_unsat@@ eq(C,n(N,!E),!H), gcd(G,!E) :- ground(C), N mod G =\= 0 | unsat(!H).
    norm1@@ gcd(G,!E) \ Ret = mul(N,!X,!E):- A = N / G, uniq(G,!X,!E) | Ret = mul(A,!X,!E).
    norm2@@ Ret = n(N,!E), gcd(G,!E) :- A = N / G | Ret = n(A,!E).
\end{lstlisting}

手順3:係数が 1, -1 となるような変数を探す.
一般性を損なわないために,$n=j, |a_j|=1$とすると,
\begin{center}
    $x_n = b/a_n - \sum_{i=1}^{n-1}a_i/a_n \:x_i$
\end{center}
で,全ての$x_n$ を右辺に置き換える.
→LMNtal だったらどうやる????

例えば,
\begin{center}
    $3x_1 + 3x_2 = 9 \wedge 2x_1 + 5x_2 = 12$
\end{center}
なら,正規化して
\begin{center}
    $x_1 + x_2 = 3 \wedge 2x_1 + 5x_2 = 12$
\end{center}
1つ目の等式を変形して
\begin{center}
    $x_1 = 3 + -1x_2$
\end{center}
だから,2つ目の式に代入して,
\begin{center}
    \begin{eqnarray}
        6 + -2x_2 + 5x_2 &=& 12\\
        3x_2 &=& 6\\
        x_2 &=& 2
    \end{eqnarray}
\end{center}
より充足


ここで、疑問.
変数を消去した後の式は,再び正規形にするべきか.
→するべき

これは,mul, mul\_n, mul\_pという3種類のアトムを用意することによって,解決.
mul になったら gcd を求めて正規化する.
mul\_n は正規化済みであることを表す(gcd アトムをそれ以上生成しないため)
elimination が起こったら,一旦 mul\_n を mul\_r に変換して,全部変換した後に mul\_p を mul に一斉に変換する
(mul に変換しながら進めると上にある gcd ルールが先に発火して大変なことになる)
elimination が起こったら token\_1 アトムが生成されて,全部変換したら token\_1 アトムが ready\_1 アトムに変換されて,gcd アトムの生成が終わったら ready\_1 アトムは削除される.

不定方程式は,上に書いたように,gcd で割り切れたら整数解を持つし,割り切れなかったら整数解を持たないので,式の conjunction が全て処理し終わった後(式が1つになったとき)に
解の判定をすれば,充足可能か決定できる.
ということで変数消去によって式を減らしていっている.

手順4:無理やり係数が 1 か -1 の式を作る\\
係数が 1, -1 のどちらかであるような変数がなくなってしまったら,
\begin{enumerate}
    \item 絶対値が最も小さい非ゼロの係数を持つ変数を選ぶ
    \item いくつかの係数が 1 か -1 になるまで繰り返し式変形を行う
    \item 係数の絶対値が 1 になった変数は先ほどのように削除する
\end{enumerate} 

symmetric mod を以下のように定義
\begin{center}
    $ a \: \widehat{mod}\: b = 
    \begin{cases}
        a \:mod\: b & :\:a \:mod\: b < b/2\\
        (a\:mod\:b)-b & :otherwise
    \end{cases}
    $
\end{center}

\begin{center}
    $m = a_n + 1 \:(a_n は全項の中で最小の係数)$
\end{center}
として,次の数式を追加.(LMNtal で実装するときは,変換後の式を基にしてアトムを追加?)
\begin{center}
    $\sum_{i=1}^n(a_i \:\widehat{mod}\: m)x_i = m \sigma + b \:\widehat{mod}\: m$
    $a_i \:\widehat{mod}\: m = -1$ より,
    $x_n = -m \sigma - b \:\widehat{mod}\:m + \sum_{i=1}^{n-1}(a_i \:\widehat{mod}\: m)x_i$
\end{center}
この最後の式を使って,他の全ての等式において,変数 $x_n$ を右辺で置き換える.
それか,decision procedures p.118 の変形を用いて(m で割るところまですっ飛ばして)以下で置き換える
\begin{center}
    $-a_n\sigma + \sum_{i=1}^{n-1}(\lfloor a_i/m + 1/2 \rfloor + (a_i \:\widehat{mod}\: m))x_i = 
    \lfloor b/m + 1/2 \rfloor + (b \:\widehat{mod}\: m)$
\end{center}
つまり,$x_n$ を表す mul\_ アトム以外の全部の mul\_ アトムの係数部分を $\lfloor a_i/m + 1/2 \rfloor + (a_i \:\widehat{mod}\: m)$
で置き換えてしまって、右辺も置き換えて,$x_n$ を表す mul\_ アトムは $-a_n\sigma$ に変えてやる

LMNtal 実装の方針としては,
\begin{enumerate}
    \item 全ての mul\_n アトムを(別の名前を付けて)複製する.この時,uniq は使えない(この手順は繰り返し行う可能性がある)ので mul\_n は何か違う名前に変えておく
    \item 複製したアトム2つにマッチして係数が小さい方を消す.(最終的に係数が一番小さい項だけ残る)
    \item $m = a_n + 1$ を表すアトムを生成する.(2番で残ったアトムを書き換える)
    \item 何か違う名前になってる mul\_n と $m=a_n+1$を表すアトムを使って全部置き換える(この時さらに違う名前にする.mul\_n に戻すとまだ危ないカモ?)
\end{enumerate}
とりあえず,ルールの依存関係を考慮して,2->3, 4->1, 1→3 になるようにすれば,直感的には4番で mul\_n に戻せそう
4->2->1→3 の順でルールを書く
→直接戻せなかったので、いいところでトークンを挿入してmul\_n に直す.


