\chapter{概要}
\section{本発表の概要}
\begin{itemize}
    \item ハイパーリンクを用いて,制約ソルバを書く.
    \item プレスバーガー算術の範囲の制約を解くためのソルバを LMNtal で実装することを目指す.
    \item まずは,自前で簡単なソルバを書いた.
    \item これからの課題:decision procedure を読むなりして, presburger arithmetic のソルバを書く.
\end{itemize}

\section{注意}
今から紹介する方法は,一般と\textcolor{red}{あまりにもかけ離れている可能性}があります.
去年の春学期に decision procedure を読んだはずですが,それも含めて知識があまりにも抜けていることに気づきました.
なので,この発表の後は,decision procedure を読んで,まともなソルバを書こうと思っています.

これからの話は,\textcolor{red}{全て間違っていると思って}聞いて,\textcolor{red}{違和感のある部分を指摘してください}.



\par


